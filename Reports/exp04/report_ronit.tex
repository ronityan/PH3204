\documentclass[12pt]{article}
\usepackage{amsmath}
\usepackage{amsfonts}
\usepackage{amssymb}
\usepackage{graphicx}
\usepackage[a4paper, margin=0.98in]{geometry}
\usepackage{physics}
\usepackage{float}
\usepackage{booktabs}
\usepackage{makecell}
\usepackage{helvet}
\usepackage{fancyhdr}
\usepackage{titling}
\usepackage{longtable}
\usepackage{caption}
\usepackage{enumitem}
\usepackage[american]{circuitikz}
\usepackage[hidelinks]{hyperref}
\usepackage{tikz}
\usepackage{subcaption}
\ctikzset{logic ports=ieee}





% Define header
\pagestyle{fancy}
\fancyhf{}
\fancyhead[R]{PH3204: Electronics Laboratory}

% Title
\title{
  \vspace{-2cm}
  \Huge \textbf{PH3204: Electronics Laboratory} \\[0.4cm]
  \Large \textbf{Experiment 04:  Study of Boolean algebra truth tables for Logic Gate functions using AND, OR, NAND,
  NOR etc. ICs}
}

\author{
  \textbf{Ronit Bhuyan (22MS025)} \\[0.2cm]
  \textbf{Sub-Group B01}
}

\date{\today}

\begin{document}


\maketitle

\tableofcontents
\noindent\rule{\textwidth}{0.4pt}
\newpage

\section{Theory}
Boolean algebra delas with variables only two possible output,$0$ and $1$ (false and true). A Boolean function takes in one or more boolean inputs and produces a bollean output. A boolean function can be implemented in the form of a boolean circuit using logic gates. Some of the most common logic gates along with their truth tables are given below:


\begin{itemize}

\item \textbf{NOT Gate}\\
  \begin{minipage}[t]{0.45\textwidth}
    \begin{figure}[H]
      \begin{center}
        \centering
      \end{center}
      \begin{circuitikz}
        %       draw a not gate 
        \draw (0,0) node[not port] (not1) {};
        \draw (not1.in) -- ++(-1,0) node[left] {$A$};
        \draw (not1.out) -- ++(1,0) node[right] {$\overline{A}$};
      \end{circuitikz}
      \caption{Symbol of NOT Gate}
      \label{fig:not}
    \end{figure}
  \end{minipage}%
  \begin{minipage}[t]{0.45\textwidth}
    \begin{table}[H]
      \centering
      %change size of the table
      \renewcommand{\arraystretch}{1.2}
      \setlength{\tabcolsep}{10pt}
      \begin{tabular}{|c||c|}
        \hline
        $A$ & $\overline{A}$ \\ \hline
        0 & 1 \\ \hline
        1 & 0 \\ \hline
      \end{tabular}
      \caption{Truth Table for NOT Gate}
      \label{tab:not}
    \end{table}
  \end{minipage}


\item \textbf{AND Gate}\\
\begin{minipage}[t]{0.45\textwidth}
\begin{figure}[H]
  \centering
  \begin{circuitikz}
    \draw (0,0) node[and port] (myport) {};
    \draw (myport.in 1) -- ++(-1,0) node[left] {$A$};
    \draw (myport.in 2) -- ++(-1,0) node[left] {$B$};
    \draw (myport.out) -- ++(1,0) node[right] {$A \cdot B$};
  \end{circuitikz}
\caption{Symbol of AND Gate}
\label{fig:and}
\end{figure}
\end{minipage}%
\begin{minipage}[t]{0.45\textwidth}
\begin{table}[H]
  \centering
  \setlength{\tabcolsep}{10pt}
  \renewcommand{\arraystretch}{1.2}
  \begin{tabular}{|c|c||c|}
    \hline
    $A$ & $B$ & $A \cdot B$ \\ \hline
    0 & 0 & 0 \\ \hline
    0 & 1 & 0 \\ \hline
    1 & 0 & 0 \\ \hline
    1 & 1 & 1 \\ \hline
\end{tabular}
\caption{Truth Table for AND Gate}
\label{tab:and}
\end{table}
\end{minipage}

\item \textbf{OR Gate}\\
\begin{minipage}[t]{0.45\textwidth}
\begin{figure}[H]
  \centering
  \begin{circuitikz}
    \draw (0,0) node[or port] (myport) {};
    \draw (myport.in 1) -- ++(-1,0) node[left] {$A$};
    \draw (myport.in 2) -- ++(-1,0) node[left] {$B$};
    \draw (myport.out) -- ++(1,0) node[right] {$A + B$};
  \end{circuitikz}
\caption{Symbol of OR Gate}
\label{fig:or}
\end{figure}
\end{minipage}%
\begin{minipage}[t]{0.45\textwidth}
\begin{table}[H]
  \centering
  \setlength{\tabcolsep}{10pt}
  \renewcommand{\arraystretch}{1.2}
  \begin{tabular}{|c|c||c|}
    \hline
    $A$ & $B$ & $A + B$ \\ \hline
    0 & 0 & 0 \\ \hline
    0 & 1 & 1 \\ \hline
    1 & 0 & 1 \\ \hline
    1 & 1 & 1 \\ \hline
  \end{tabular}
\caption{Truth Table for OR Gate}
\label{tab:or}
\end{table}
\end{minipage}

\item \textbf{NAND Gate}\\
\begin{minipage}[t]{0.45\textwidth}
\begin{figure}[H]
  \centering
  \begin{circuitikz}
    \draw (0,0) node[nand port] (myport) {};
    \draw (myport.in 1) -- ++(-1,0) node[left] {$A$};
    \draw (myport.in 2) -- ++(-1,0) node[left] {$B$};
    \draw (myport.out) -- ++(1,0) node[right] {$\overline{A \cdot B}$};
  \end{circuitikz}
\caption{Symbol of NAND Gate}
\label{fig:nand}
\end{figure}
\end{minipage}%
\begin{minipage}[t]{0.45\textwidth}
\begin{table}[H]
  \centering
  \setlength{\tabcolsep}{10pt}
  \renewcommand{\arraystretch}{1.2}
  \begin{tabular}{|c|c||c|}
  \hline
  $A$ & $B$ & $\overline{A \cdot B}$ \\ \hline
  0 & 0 & 1 \\ \hline
  0 & 1 & 1 \\ \hline
  1 & 0 & 1 \\ \hline
  1 & 1 & 0 \\ \hline
  \end{tabular}
  \caption{Truth Table for NAND Gate}
  \label{tab:nand}
\end{table}
\end{minipage}

\item \textbf{NOR Gate}\\
\begin{minipage}[t]{0.45\textwidth}
\begin{figure}[H]
  \centering
  \begin{circuitikz}
    \draw (0,0) node[nor port] (myport) {};
    \draw (myport.in 1) -- ++(-1,0) node[left] {$A$};
    \draw (myport.in 2) -- ++(-1,0) node[left] {$B$};
    \draw (myport.out) -- ++(1,0) node[right] {$\overline{A + B}$};
  \end{circuitikz}
\caption{Symbol of NOR Gate}
\label{fig:nor}
\end{figure}
\end{minipage}%
\begin{minipage}[t]{0.45\textwidth}
\begin{table}[H]
  \centering
  \setlength{\tabcolsep}{10pt}
  \renewcommand{\arraystretch}{1.2}
  \begin{tabular}{|c|c||c|}
  \hline
  $A$ & $B$ & $\overline{A + B}$ \\ \hline
  0 & 0 & 1 \\ \hline
  0 & 1 & 0 \\ \hline
  1 & 0 & 0 \\ \hline
  1 & 1 & 0 \\ \hline
  \end{tabular}
  \caption{Truth Table for NOR Gate}
  \label{tab:nor}
\end{table}

\end{minipage}

\item \textbf{XOR Gate}\\
\begin{minipage}[t]{0.45\textwidth}
\begin{figure}[H]
  \centering
  \begin{circuitikz}
    \draw (0,0) node[xor port] (myport) {};
    \draw (myport.in 1) -- ++(-1,0) node[left] {$A$};
    \draw (myport.in 2) -- ++(-1,0) node[left] {$B$};
    \draw (myport.out) -- ++(1,0) node[right] {$A \oplus B$};
  \end{circuitikz}
\caption{Symbol of XOR Gate}
\label{fig:xor}
\end{figure}
\end{minipage}%
\begin{minipage}[t]{0.45\textwidth}
  \begin{table}[H]
    \centering
    \setlength{\tabcolsep}{10pt}
    \renewcommand{\arraystretch}{1.2}
    \begin{tabular}{|c|c||c|}
      \hline
      $A$ & $B$ & $A \oplus B$ \\ \hline
      0 & 0 & 0 \\ \hline
      0 & 1 & 1 \\ \hline
      1 & 0 & 1 \\ \hline
      1 & 1 & 0 \\ \hline
    \end{tabular}
    \caption{Truth Table for XOR Gate}
    \label{tab:xor}
  \end{table}
\end{minipage}
\end{itemize}
In our experiment, we have used ICs for implementing the above mentioned logic gates. The ICs used are listed below:
\begin{itemize}
  \item \textbf{IC7400}: NAND Gate
  \item \textbf{IC7402}: NOR Gate
  \item \textbf{IC7404}: NOT Gate
  \item \textbf{IC7408}: AND Gate
  \item \textbf{IC7432}: OR Gate
  \item \textbf{IC7486}: XOR Gate
\end{itemize}

\section{Boolean Circuit Verification}
We shall verify three different boolean circuits in this experiment. The circuitds are drawn with using ICs, bread boards and the output is observed using an LED Bulb. The glowing of the LED indicates an output $1$ while $0$ is indicated by the LED not glowing. 
\newpage
\subsection{Example 1}
The first example boolean circuit is drawn below.
\begin{figure}[H]
  \centering
  \begin{circuitikz}
    \node[and port] (and1) at (0,0) {};
    \draw (and1.in 1) -- ++(-1,0) node[left] {$A$};
    \draw (and1.in 2) -- ++(-1,0) node[left] {$B$};

    \node[nor port] (nor1) at (5,-0.25) {};

    \node[nand port] (nand1) at (2.5,-2) {};

    \node[or port] (or1) at (0,-5) {};
    \draw (or1.in 1) -- ++(-1,0) node[left] {$C$};
    \draw (or1.in 2) -- ++(-1,0) node[left] {$D$};

    \draw (or1.out) -- ++(1,0) -| (nand1.in 2);
    \draw (and1.in 2) -- ++(0,-1) -| (nand1.in 1);
    \draw (and1.out) -- ++(1,0) -- (nor1.in 1);

    \node[xor port] (xor) at (6,-5.25) {};

    \node[and port] (and2) at (9,-2) {};

    \draw (or1.out)  -- ++(1,0) |- (xor.in 2);

    \draw (nand1.out) -- ++(0.25,0) -| (xor.in 1);

    \draw (nor1.out) -- ++(0.25,0) -| (and2.in 1);
    \draw (xor.out) -- ++(0.5,0) -| (and2.in 2);
    \draw (and2.out) -- ++(1,0) node[right] {$Q$};

    \draw (nand1.out) -- ++(0.25,0) -| (nor1.in 2);

  \end{circuitikz}
\caption{Boolean Circuit for Example 1}
\label{fig:example1}
\end{figure}
\setlength{\tabcolsep}{10pt}
\renewcommand{\arraystretch}{1.2}
\noindent
The theoretical boolean expression of the circuit is given by:
\begin{equation*}
\boxed{
    \text{Q}=\overline{\text{A}}\text{B}.(\text{C}+\text{D})
}
\end{equation*}
\noindent
Experimentally, the truth table for the circuit drawn above has been tabulated below
\begin{longtable}{|c|c|c|c||c|}
	\hline
    $\mathrm{A}$  & $\mathrm{B}$ & $\mathrm{C}$ & $\mathrm{D}$ & $\mathrm{Q}$ \\
  \hline
	\endfirsthead
	\hline
   $\mathrm{A}$  & $\mathrm{B}$ & $\mathrm{C}$ & $\mathrm{D}$ & $\mathrm{Q}$ \\ \hline
	\endhead
	\hline
	\endfoot
	
  \endlastfoot
  0 & 0 & 0 & 0 & 0 \\ \hline
  0 & 0 & 0 & 1 & 0 \\ \hline
  0 & 0 & 1 & 0 & 0 \\ \hline
  0 & 0 & 1 & 1 & 0 \\ \hline
  0 & 1 & 0 & 0 & 0 \\ \hline
  0 & 1 & 0 & 1 & 1 \\ \hline
  0 & 1 & 1 & 0 & 1 \\ \hline
  0 & 1 & 1 & 1 & 1 \\ \hline
  1 & 0 & 0 & 0 & 0 \\ \hline
  1 & 0 & 0 & 1 & 0 \\ \hline
  1 & 0 & 1 & 0 & 0 \\ \hline
  1 & 0 & 1 & 1 & 0 \\ \hline
  1 & 1 & 0 & 0 & 0 \\ \hline
  1 & 1 & 0 & 1 & 0 \\ \hline
  1 & 1 & 1 & 0 & 0 \\ \hline
  1 & 1 & 1 & 1 & 0 \\ \hline
  \caption{Truth Table for Example 1}
  \end{longtable}
\noindent
In each of the cases , we observed that the experimental output of the circuit matched with what was expexted from the theoretical expression. The truth table for the circuit was hence verified to be correct.
\subsection{Example 2}
The second boolean circuit that we shall verify is given below:
\begin{figure}[H]
  \centering
  \begin{circuitikz}
    \node[and port] (and1) at (0,0) {};
    \draw (and1.in 1) -- ++(-1,0) node[left] {$A$};
    \draw (and1.in 2) -- ++(-1,0) node[left] {$B$};

    \node[nor port] (nor1) at (5,-0.25) {};

    \node[xor port] (xor) at (2.5,-2) {};

    \node[nand port] (nand) at (0,-5) {};
    \draw (nand.in 1) -- ++(-1,0) node[left] {$C$};
    \draw (nand.in 2) -- ++(-1,0) node[left] {$D$};

    \draw (nand.out) -- ++(1,0) -| (xor.in 2);
    \draw (and1.in 2) -- ++(0,-1) -| (xor.in 1);
    \draw (and1.out) -- ++(1,0) -- (nor1.in 1);

    \node[and port] (AND) at (6,-5.25) {};

    \node[or port] (or) at (9,-2) {};

    \draw (nand.out)  -- ++(1,0) |- (AND.in 2);

    \draw (xor.out) -- ++(0.25,0) -| (AND.in 1);

    \draw (nor1.out) -- ++(0.25,0) -| (or.in 1);
    \draw (AND.out) -- ++(0.5,0) -| (or.in 2);
    \draw (or.out) -- ++(1,0) node[right] {$Q$};

    \draw (nand1.out) -- ++(0.25,0) -| (nor1.in 2);

  \end{circuitikz}
\caption{Boolean Circuit for Example 2}
\label{fig:example2}
\end{figure}
\noindent
The theoretical boolean expression of the circuit is given by:
\begin{equation*}
\boxed{
    \text{Q}=\overline{\text{A}}.(\overline{\text{C}.\text{D}})+\overline{\text{B}}
}
\end{equation*}
\noindent
Experimentally, the truth table for the circuit drawn above has been tabulated below:
\setlength{\tabcolsep}{10pt}
\renewcommand{\arraystretch}{1.2}
\begin{longtable}{|c|c|c|c||c|}
	\hline
    $\mathrm{A}$  & $\mathrm{B}$ & $\mathrm{C}$ & $\mathrm{D}$ & $\mathrm{Q}$ \\
  \hline
	\endfirsthead
	\hline
   $\mathrm{A}$  & $\mathrm{B}$ & $\mathrm{C}$ & $\mathrm{D}$ & $\mathrm{Q}$ \\ \hline
	\endhead
	\hline
	\endfoot
	
  \endlastfoot
  0 & 0 & 0 & 0 & 1 \\ \hline
    0 & 0 & 0 & 1 & 1 \\  \hline
    0 & 0 & 1 & 0 & 1 \\  \hline
    0 & 0 & 1 & 1 & 1 \\  \hline
    0 & 1 & 0 & 0 & 1 \\  \hline
    0 & 1 & 0 & 1 & 1 \\  \hline
    0 & 1 & 1 & 0 & 1 \\  \hline
    0 & 1 & 1 & 1 & 0 \\  \hline
    1 & 0 & 0 & 0 & 1 \\  \hline
    1 & 0 & 0 & 1 & 1 \\  \hline
    1 & 0 & 1 & 0 & 1 \\  \hline
    1 & 0 & 1 & 1 & 1 \\  \hline
    1 & 1 & 0 & 0 & 1 \\  \hline
    1 & 1 & 0 & 1 & 0 \\  \hline
    1 & 1 & 1 & 0 & 0 \\  \hline
    1 & 1 & 1 & 1 & 0 \\  \hline
  \caption{Truth Table for Example 2}
  \end{longtable}
\noindent
In each of the cases , we observed that the experimental output of the circuit matched with what was expexted from the theoretical expression. The truth table for the circuit was hence verified to be correct.

\subsection{Example 3}
The final example that we shall verify is shown below:
\begin{figure}[H]
  \centering
  \begin{circuitikz}
    \node[nor port] (nor) at (0,0) {};
    \draw (nor.in 1) -- ++(-1,0) node[left] {$A$};
    \draw (nor.in 2) -- ++(-1,0) node[left] {$B$};

    \node[nand port] (nand) at (5,-0.25) {};

    \node[or port] (or) at (2.5,-2) {};

    \node[and port] (and) at (0,-5) {};
    \draw (and.in 1) -- ++(-1,0) node[left] {$C$};
    \draw (and.in 2) -- ++(-1,0) node[left] {$D$};

    \draw (and.out) -- ++(1,0) -| (or.in 2);
    \draw (nor.in 2) -- ++(0,-1) -| (or.in 1);
    \draw (nor.out) -- ++(1,0) -- (nand.in 1);

    \node[and port] (and3) at (6,-5.25) {};

    \node[xor port] (xor2) at (9,-2) {};

    \draw (or1.out)  -- ++(1,0) |- (and3.in 2);

    \draw (or.out) -- ++(0.25,0) -| (and3.in 1);

    \draw (nand.out) -- ++(0.25,0) -| (xor2.in 1);
    \draw (and3.out) -- ++(0.5,0) -| (xor2.in 2);
    \draw (xor2.out) -- ++(1,0) node[right] {$Q$};

    \draw (or.out) -- ++(0.25,0) -| (nand.in 2);

  \end{circuitikz}
\caption{Boolean Circuit for Example 3}
\label{fig:example3}
\end{figure}
\noindent
\begin{equation*}
  \boxed{
    \text{Q}=\overline{A}.\overline{B}+\overline{C}+\overline{D}
  }
\end{equation*}
Exoerimentally, the truth table for the circuit drawn above has been tabulated below:
\setlength{\tabcolsep}{10pt}
\renewcommand{\arraystretch}{1.2}
\begin{longtable}{|c|c|c|c||c|}
	\hline
    $\mathrm{A}$  & $\mathrm{B}$ & $\mathrm{C}$ & $\mathrm{D}$ & $\mathrm{Q}$ \\
  \hline
	\endfirsthead
	\hline
   $\mathrm{A}$  & $\mathrm{B}$ & $\mathrm{C}$ & $\mathrm{D}$ & $\mathrm{Q}$ \\ \hline
	\endhead
	\hline
	\endfoot
	
  \endlastfoot
  0 & 0 & 0 & 0 & 1 \\ \hline
    0 & 0 & 0 & 1 & 1 \\ \hline
    0 & 0 & 1 & 0 & 1 \\ \hline
    0 & 0 & 1 & 1 & 1 \\ \hline
    0 & 1 & 0 & 0 & 1 \\ \hline
    0 & 1 & 0 & 1 & 1 \\ \hline
    0 & 1 & 1 & 0 & 1 \\ \hline
    0 & 1 & 1 & 1 & 0 \\ \hline
    1 & 0 & 0 & 0 & 1 \\ \hline
    1 & 0 & 0 & 1 & 1 \\ \hline
    1 & 0 & 1 & 0 & 1 \\ \hline
    1 & 0 & 1 & 1 & 0 \\ \hline
    1 & 1 & 0 & 0 & 1 \\ \hline
    1 & 1 & 0 & 1 & 1 \\ \hline
    1 & 1 & 1 & 0 & 1 \\ \hline
    1 & 1 & 1 & 1 & 0 \\ \hline
  \caption{Truth Table for Example 3}
  \end{longtable}
\noindent
In each of the cases , we observed that the experimental output of the circuit matched with what was expexted from the theoretical expression. The truth table for this circuit was hence verified to be correct as well.

\section{Conclusion}
In this experiment, we studied three different boolean circuits made from a combination of different logic gates and verified the truth tables for each of the circuit. In each case , the truth tables were verified to be correct.

\section{Sources of Error}
The following sources of error were observed during the experiment:
\begin{itemize}
  \item The connections made on the bread board may not very stable and hence could give rise to incorrect results. Errors may also occur due to disfunctioning of the ICs used.
  \item The LED Bulb used to indicate the output maybe not be bright enough. This could cause difficulty and errors in observing the output.
\end{itemize}

\end{document}