\documentclass[12pt]{article}
\usepackage{amsmath}
\usepackage{amsfonts}
\usepackage{amssymb}
\usepackage{graphicx}
\usepackage[a4paper, margin=0.98in]{geometry}
\usepackage{physics}
\usepackage{float}
\usepackage{booktabs}
\usepackage{makecell}
\usepackage{helvet}
\usepackage{fancyhdr}
\usepackage{titling}
\usepackage{longtable}
\usepackage{caption}
\usepackage{enumitem}
\usepackage{circuitikz}
\usepackage[hidelinks]{hyperref}

% Define header
\pagestyle{fancy}
\fancyhf{}
\fancyhead[R]{PH3204: Electronics Laboratory}

% Title
\title{
  \vspace{-2cm}
  \Huge \textbf{PH3204: Electronics Laboratory} \\[0.4cm]
  \Large \textbf{Experiment 02: Study of characteristics of an n-p-n bipolar junction transistor (BJT)}
}

\author{
  \textbf{Ronit Bhuyan (22MS025)} \\[0.2cm]
  \textbf{Sub-Group B01}
}

\date{\today}

\begin{document}

\maketitle

\tableofcontents
\noindent\rule{\textwidth}{0.4pt}
\newpage

\section{Theory}
\subsection{Bipolar Juction Transistor (BJT)}
A \textbf{Bipolar Junction Transistor (BJT)} is a three-terminal device having wide applications in the construction of amplifiers, gates, oscillators, etc. A BJT comprises three doped semiconductor regions, namely \textbf{emitter, base, and collector}. 
There are two p-n junctions in a transistor, one between the emitter and base and the other between the base and collector.The emitter is heavily doped, the base is lightly doped, and the collector is moderately doped. The emitter-base junction is forward-biased, and the base-collector junction is reverse-biased.\\
The BJT is classified into two types based on the majority charge carriers in the three regions, namely \textbf{n-p-n transistor }and \textbf{p-n-p transistor.}
\begin{itemize}
  \item \textbf{n-p-n Transistor:} In an n-p-n transistor, the emitter and collector are doped with n-type impurities while the base is doped with p-type impurities.
  \item \textbf{p-n-p Transistor:} In a p-n-p transistor, the emitter and collector are doped with p-type impurities while the base is doped with n-type impurities.
\end{itemize}
Since, there are three terminals in a transistor, there exists three possible of operations, where one of the three terminals is common to both Input and Output. These modes are \textbf{Common Emitter(CE) Configuration, Common Base(CB) Configuration}, and \textbf{Common Collector(CC) Configuration}. Out of these three, the Common Emitter Configuration is the most widely used one.\\
\\
In this experiment, we shall study the Input and Output Characteristics of an n-p-n BJT in the Common Emitter Configuration.

\subsection{Common Emitter (CE) Configuration}
Due to its high current,voltage and power gain, the Common Emitter Configuration is the most widely used configuration. The input is applied between the base and emitter, while the output is taken between the collector and emitter. The emitter is common to both input and output. The input characteristics of a CE configuration is the plot of the input current $I_B$ versus the input voltage $V_{BE}$, and the output characteristics is the plot of the output current $I_C$ versus the output voltage $V_{CE}$.
\begin{figure}[H]    
  \centering
  \begin{circuitikz}[american voltages ]
      \draw
      % Transistor
      (0,0) node[npn, anchor=E, tr circle] (Q) {}
      (Q.B) node[above left] {B}
      (Q.C) node[right] {C}
      (Q.E) node[below left] {E}
      % Base resistor
      (Q.B) -- (-1,0.75) to[R, l=$\mathrm{R_B}$] (-3.5,0.75) to [rmeter, t = $\mathrm{\mu A}$]  (-5,0.75)
      to[battery, v_=$\mathrm{V_{BB}}$] (-5,-2) -- (0,-2)
  
      % Collector resistor
      (Q.C) -- (0,2) to[R, l=$\mathrm{R_C}$] (3,2)
      to [rmeter, t= $\mathrm{mA}$] (4,2) -- (5,2)
      to [battery, v=$\mathrm{V_{CC}}$] (5,-2) -- (0,-2) 
  
      % Ground connection
      (Q.E) -- (0,-2) node[ground] {};
  \end{circuitikz}   
  \caption{Circuit diagram of a n-p-n BJT in CE Configuration}  
\end{figure}
\section{CE Input Characteristics}

\section{CE Output Characteristics}
\section{Conclusion}
\section{Sources of Error}
 
\end{document}